\chapter{Background}

Fiber-reinforced composite materials are seeing a widespread use in a large number of applications ranging from aerospace systems over renewable energy production to automotive parts \cite{park2011interface} \cite{HighPerformanceTextiles}. These composite materials generally provide high specific strength and improved stiffness compared to other materials \cite{GeneralizedContinuumMechanics}. Through the combination of different materials, desirable mechanical properties like low weight with high stiffness can be achieved that would be hard or near impossible to recreate with single compound materials \cite{AdvancesDamageMechanics}. An FRC can generally be described as the combination of three components \cite{AdvancedDentalBiomaterials}: \\
\begin{itemize}
	\item The matrix, made of a polymer. This polymer can either be applied as a resin and hardens irreversible or a thermoplastic is used, that needs to be heated for application.
	\item The reinforcement component which consists of fibers with high strength and modulus. These days preferred materials are glass, carbon or polyethylene fibers.
	\item The fine interphase region. It's the interface between the matrix and the reinforcement that transfers the load between these.	
\end{itemize}
%
FRC's can look back to a long history since the beginning of the 19th century with phenolic sheet Bakelite being the first fiber-reinforced plastic. Bakelite, a thermoset being the matrix was combined with different fiber materials like paper, cotton fabrics, or synthetic fabrics to create parts that can meet diverse mechanical, electrical and thermal requirements.\cite{BakelitePhenolics}\\
\\
In this project, a thermoplastic matrix will be used together with carbon or glass fibers. The mixed reinforced thermoplastic being supplied as pellets in various sizes for different properties is dried and then melted in an extruder through a continuous process. The material is then portioned into pieces by a guillotine into pieces and combined into stacks and assemblies by a delta robot. These assemblies are then brought to temperature again before being loaded into a hydraulic press with a mold by an industrial robot arm.\cite{SystemRequirements}





