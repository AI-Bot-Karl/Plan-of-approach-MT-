% !TeX spellcheck = <none>

\chapter{Requirement Analysis}

Certain conditions need to be fulfilled to make the project a full full success. 
These can be split up into external and internal factors.\\
\\
\underline{\textbf{External requirements:}}\vspace{2mm}
\begin{itemize}[leftmargin=5cm]
	\item[\textbf{Power Supply}] In order to supply the whole production line with enough energy, 400A fuses need to be installed in the 400V 3-Phase building connection.
	\item[\textbf{Surrounding Subprocesses}] All subprocesses (extruder with supply hopper, conveyors, delta robot with gripper, press with mould, robot gripper, oven with heat-shield) need to be in place and running on their own.
	\item[\textbf{Test Material}] Some kind of test material needs to be supplied to demonstrate all abilities of the robot arm.
\end{itemize}
\bigskip
\underline{\textbf{Internal requirements:}}\vspace{2mm}
\begin{itemize}[leftmargin=5cm]
	\item[\textbf{Fieldbus}] An Ethernet network needs to be set up to host the communication via Profinet between the subprocesses.
	\item[\textbf{Software}] For simulation Visual Components and for control ROBOGUIDE with Karol Programming Language as well as TIA Portal need to be made available, installed and configured. 
	\item[\textbf{Robot Placement}] The final placement of the robot is not yet determined and the robot needs to be bolted to the ground in its final position before it can be used.
\end{itemize}
\bigskip

