% !TeX spellcheck = en_GB

\chapter{Restrictions}

\begin{itemize}[leftmargin=5cm]
	\item[\textbf{Surrounding subprocesses}] To fully demonstrate all abilities of the  FANUC R-2000iC/210F, all sub-processes (extruder with supply hopper, conveyors, delta robot with gripper, press with mould, robot gripper, oven with heat-shield) need to be in place and running on their own. Also test material will be needed that can be picked up and placed.
	\item[\textbf{Gripper}] The robot arm currently does not have a gripper. To demonstrate the picking abilities in real life with test material, this gripper will need to be designed in time.
	\item[\textbf{Temperature range of material}] For the Process to operate correctly, the material taken out of the oven needs to be hot enough to stay formable. This will also be a core point of the overall research to find an ideal oven-temperature in the range between 230-330°C.
	\item[\textbf{Fieldbus}] An Ethernet network needs to be set up to host the communication via Profinet between the subprocesses.
	\item[\textbf{Software}] For simulation Visual Components and for control ROBOGUIDE with Karol Programming Language as well as TIA Portal need to be made available, installed and configured. 
	\item[\textbf{Robot Placement}] The final placement of the robot is not yet determined and the robot needs to be bolted to the ground in its final position before it can be used.
\end{itemize}