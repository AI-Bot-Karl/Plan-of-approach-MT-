% !TeX spellcheck = en_GB

\chapter{Master Level}

This Thesis project demands creating a model of the robotic arm. The modelling of the 6 axis FANUC R-2000iC/210F will require knowledge from Systems Modelling. If parameters cannot be determined, they can be found with techniques from Systems Identification. 
Based on this model, a controller can be developed and a range of paths can be found through inverse kinematics. This controller is an extension of the subjects taught in the module Advanced Controller Design. 
The number of resulting paths can then be narrowed down further, for example based on safety restrictions or the desired working area of the robot. This will be a further development of the knowledge conveyed in Applied Control Strategy from the module Applied Control. 
This results in a control scheme that can be implemented in Matlab and simulated numerically, and probably also graphically. The basics of implementation were shown in Controller Implementation from the module Applied Control

I will demonstrate my master level by understanding and simulating the dynamics of a 6 axis Robot arm. 

%Literature Research on advanced control strategies for better proof of master level

%This results in a control scheme that can be implemented in the Karel programming language on the R-30iB controller. The basics of implementation were shown in Controller Implementation from the module Applied Control, although different programming languages will be used here. 
%The control logic blending the robot into the overall process will be implemented on the Simatic S7-1500 PLC using the TIA portal software. This combination of multiple controllers working together was taught in the module Distributed Systems.\\
%\\
%Besides this systematic approach for control, other tasks will be carried out. This includes the whole process from setting up the robot for first operation down to fine tuning for maximum efficiency. Additional requirements like safety need to be taken into account besides the goals of control. On top of that, feedback control with corrective signals from other subprocesses needs to be developed for optimal efficiency.\\
%\\
%This project requires all abilities of a Control Systems Engineer, as not only knowledge taught in class is applied, but needs to be extrapolated with external knowledge, e.g. from scientific papers, in order to work with the given system. Multi-domain insight brought together in one control strategy will result in a smooth and reliable production process. Finally, as this project is a team effort where all knowledge is shared and implemented where required, efforts in team lead and management need to be made.\\
%\\
%With this wholistic approach I will demonstrate my master level.
%
%%%learning new software languages 
